Прогнозирование социально-экономических процессов в целом и наркоситуации в
частности --- один из наиболее важных инструментов для поддержки принятия
решений при управлении региональным развитием, позволяя планировать действия и
ресурсы  в соответствии с гипотетическим состонием социальной системы в будущем.

Немаловажным обстоятельством является включенность системы
информационно-аналитической поддержки контроля за наркоситуацией в
общегосударственную стратегию антинаркотической политики, что требует строгой
подчиненности данной системы целям и задачам, выдвигаемым на данном этапе
реализации антинаркотической политики. Поэтому применяемые методы должны
максимально эффективно обеспечивать достижение целей мониторинга и анализа
наркоситуации.

Основные теоретические результаты работы носят следующий характер: 
\begin{itemize}
\item Изучены используемые в мировой практике антинаркотические политики;
\item Изучены методы и модели прогнозирования временных рядов;
\item Проведено сравнение моделей на предмет их эффективности в региональном
управлении;
\item Сформулированы требования к требуемой модели прогнозирования численности
наркозависимых на территории Санкт-Петербурга;
\item Спроектирован программный модуль прогнозирования для
информационно-аналитической системы.
\end{itemize}

В области практических результатов удалось достичь следующего:
\begin{itemize}
\item Модифицирована модель прогнозирования на основе нечеткой логики под нужды
анализа наркоситуации;
\item Разработан модуль прогнозирования для информационно-аналитической системы;
\item Разработаны рекомендации к применению новой модели и 
его программного инструмента;
\item Продемонстрированы характеристики интеллектуального анализа данных как
средства регионального управления.
\end{itemize}

В связи с тем, что с каждым годом наркоситуация сопровождается всё новыми
вызовами, одними из последних для примера можно назвать распространение
синтетических наркотиков, требуются ответы на эти вызовы. В том числе это
качается подсистемы информационно-аналитического обеспечения принятия решений.
Разработанная модель и сопутствующий  программный инструмент в силу своей
универсальности могут быть не только использованы для решения своей прямой
задачи --- прогнозирования численности наркозависимых, но и могут быть
адаптированы для других задач, в особенности имеющих отношение к сложным,
многосоставным связям между явлениями.

Разработанный инструмент, учитывая рекомендуемые условия его применения,  можно
рекомендовать к внедрению в качестве компонента информационно-аналитической
системы. Следует отметить обнаружившееся в ходе работы обстоятельство наличия у
интеллектуальных методов прогнозирования специфических аналиитических продуктов,
удобных интсрументов анализа которых пока что не обнаружено. В случае нечеткой
логики таким продуктов является база нечетких правил. Таким образом, можно
утверждать о необходимости развития области интеллектуального моделирования и
прогнозирования для наилучшего применения теоретических достижений на практике.
