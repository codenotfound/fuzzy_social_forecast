\section*{Введение}
\addcontentsline{toc}{section}{Введение}

Наркоситуация как разновидность социальной ситуации представляет собой 
ограниченную временными и пространственными рамками совокупность социальных 
процессов, складывающихся в результате взаимодействия различных сторон 
общественных отношений.  Рассматривая наркоситуацию как разновидность 
социального взаимодействия, важно в первую очередь использовать системный 
подход, выделяя элементы, составляющие наркоситуацию как систему, а также 
интерпретируя характер взаимодействия между ними.

Полноценное моделирование наркоситуации, по нашему мнению, должно учитывать 
взаимовлияние социально-экономических индикаторов, описывающих наркоситуацию, 
тем самым оно может представлять интуитивно понятную логику протекания 
процессов, и выявлять необходимые для изменения ситуации рычаги управления.  
Немалую роль здесь играет и сам характер доминирующей в стране системы 
управления в области противодействия наркомании.  В данной главе 
рассматривается процесс антинаркотического регулирования <<сверху-вниз>> с тем, 
чтобы согласовать прикладное моделирование с целями, выдвигаемыми государством 
при реализации стратегий развития наркоситуации.

\textbf{Цель главы:} разработка модели прогнозирования доли наркозависимых в 
населении Санкт-Петербурга с помощью индикаторов наркотизации.

\textbf{Поставленные задачи} для достижения цели данной главы:
\begin{itemize}
    \item обзор существующих антинаркотических политик и способов их 
        информационной поддержки.
\end{itemize} 



\newpage
\section{Международный опыт использования информационных систем при реализации 
    государственной антинаркотической политики}

Обеспечение национальной безопасности — одна из важнейших функций государства.  
В Стратегии национальной безопасности Российской Федерации 
\cite{ru_nat_def_strat} наркомания определяется как социально значимое 
заболевание, требующее разработки «единых общероссийских подходов к диагностике, 
лечению и реабилитации пациентов».  Снижение уровня наркомании должно 
способствовать «повышению качества жизни российских граждан».

Однако, существующие модели организации антинаркотической деятельности далеко не 
всегда достигают цели, а бюджетные средства, которые инвестируются в их 
реализацию, не окупаются. Поэтому для исследования актуально обратиться к 
накопленному международному опыту. Предмет данной главы — соотношение между 
антинаркотической политикой государства и информационными системами, призванными 
обеспечить её выполнение. В данной главе приводится обзор основных видов 
наркополитик, обзор и классификация некоторых из существующих в мире 
информационных систем, обеспечивающих противодействие распространению наркомании 
в обществе, а также применяемых в аналитике наркоситуации методов
прогнозирования.

\subsection{Виды антинаркотических политик}

Рассмотрим виды наркополитик. Политика \textbf{снижения потребления} (use 
reduction) нацелена на устранение, или по меньшей мере снижение употребления 
наркотиков в обществе. Уменьшение потребления —  основная цель контроля над 
наркотиками в США, что отражено в Национальной стратегии по борьбе с наркотиками 
(National Drug Control Strategy) \cite{us_nat_drug_strat}. Она также занимает 
видное место в антинаркотических политиках европейских стран, таких как Швеция и 
Франция, а также странах Ближнего Востока. Политика уменьшения потребления 
основывается на мысли, что проблемы, связанные с наркотиками,  возникающие в 
обществе, семье, социальных группах, могут быть решены, только если прекратить 
или минимизировать употребление наркотиков. 

Для людей, которые не употребляют наркотики, парадигма уменьшения потребления 
предполагает профилактические программы, разработанные с намерением 
предотвратить употребление наркотиков.

Политика \textbf{снижения вреда на микроуровне} (micro harm reduction) нацелена 
на снижение среднего вреда отдельным наркозависимым и лицам, не употребляющим 
наркотики. Эта политика опирается на мысль, что само по себе употребление 
наркотиков лишь умеренно опасно, и можно предпринять шаги для уменьшения рисков, 
сопутствующих употреблению наркотиков. Термин «снижение вреда» может 
трактоваться широко, однако эта гибкость может стать источником путаницы, давая 
возможность ввода в действие двух диаметрально противоположных политик, 
направленных при этом одну цель — снижение вреда (например, программы обмена 
шприцев для наркозависимых с целью уменьшения распространения инфекционных 
заболеваний и, следовательно, снижения вреда для наркозависимых, и обязательное 
лишение свободы для лиц, употребляющих наркотики, с целью снижения вреда для 
тех, кто не употребляет). Поэтому при анализе нужно четко определять задачи, 
поскольку есть различные негативные эффекты, связанные с употреблением 
наркотиков, каждый со своим контекстом употребления и противодействия 
наркотикам.

Среди стран, придерживающихся данного подхода, можно выделить Швейцарию, 
Нидерланды, Канаду.
 
Снижение вреда на микроуровне уходит корнями как в общественное здравоохранение, 
так и в более широкое движение за нормализацию употребления наркотиков. 
Внедрение современных методов снижения вреда в населенных пунктах США является 
следствием более ранних европейских опытов с программами обмена шприцев и 
контролируемого распространения наркотиков. 

Мероприятия по реализации антинаркотической политики могут оказывать влияние на 
распространенность, интенсивность и вред от употребления наркотиков. Понятие 
тотального вреда (total harm reduction) охватывает все эти явления. Политика 
\textbf{тотального снижения вреда} объединяет черты снижения потребления и 
снижения вреда на микроуровне \cite{MacCoun2001}. Макровред определяется как 
произведение  распространенности, интенсивности и среднего вреда от наркотиков. 
Тотальный вред определяется как сумма макровредов. Данный подход тесно связан с 
анализом рисков, экономического ущерба и т. д. На данный момент он существует, 
прежде всего, как теоретическая модель. 

С практической точки зрения, можно выстроить антинаркотические мероприятия от 
более строгих (снижение употребления) к менее строгим (снижение вреда): 
полицейский надзор за уличной торговлей наркотиками, полицейские «облавы», 
ограничение оборота прекурсоров, арест за малые нарушения, в т.ч. выращивание 
марихуаны, тестирование на наркотики, вмешательство в частную жизнь 
наркозависимых, их обучение, наркосуды, лечение и реабилитация.

\subsection{Использование информационных систем при реализации антинаркотической 
    политики}

Рассмотрим применение информационных систем при реализации антинаркотической 
политики на примере США, придерживающейся преимущественно политики снижения 
потребления, Европы, некоторые из стран которой придерживаются политики
снижения вреда, и ряда других стран. 

Кратко опишем структуру государственного управления в области контроля за
оборотом наркотиков и формирования наркополитики в США. Главным лицом,
формирующим наркополитику в США (после Президента), является т.н. <<Drug czar>>,
глава ONDCP (Управления национальной политики по контролю за наркотиками),
который отчитывается перед Конгрессом. К функциям ONDCP относится: создание
координированной национальной стратегии противодействия наркотикам, создание
центра управления антинаркотическими проектами и контроль антинаркотических
бюджетов.  Из правоохранительных органов главную роль в борьбе с контрабандой и
употреблением нароктиков играет DEA (Управление по контролю за распространением
наркотиков). В области исследований следует выделить как научные учреждения
(Комитет по данным и исследованиям для политики в области незаконных наркотиков
в составе Национальной академии наук), так и частные аналитические агентства
(RAND и др.) \cite{Robinson2007,InfAmerPolicy2001}.

К категории мониторинговых систем можно отнести PDMP, созданную для 
идентификации лиц, злоупотребляющих наркосодержащими лекарствами, ограничения 
выписывания и продажи данной категории лекарств как государственными, так и 
частными клиниками и аптеками. В эту же категорию входят опросы, такие как ADAM 
(тестирование арестантов на наличие следов приема наркотиков), NSDUH, SAMHDA 
(общенациональные опросы),  National Roadside Survey (тестирование водителей на 
алкоголь и наркотики). 

NADDIS — это система сбора и индексирования данных, содержащая миллионы личных 
дел граждан, для доступа полиции и наркоаналитиков. ADNET — ИС Департамента 
обороны, функции которой — мониторинг, обеспечение мероприятий по снижению 
спроса на наркотики, обеспечение правоприменения как внутри, так и вне страны. 
Проект HIDTA нацелен на координацию деятельности различных государственных 
агентств в т.н. «высокоинтенсивных зонах», в частности на границе с Мексикой, с 
целью пресечения наркотрафика. STRIDE — система извлечения информации о
результатах лабораторного анализа образцов наркотиков из материалов уголовных 
дел по наркопреступлениям.

В Европейском союзе функционирует децентрализованная организация European 
Monitoring Centre for Drugs and Drug Addiction (EMCDDA), которая ежегодно 
публикует отчёты о состоянии наркотических проблем в государствах-членах 
Евросоюза, собирает и предоставляет актуальные эмпирические данные для как для 
учёных, так и для политиков. Деятельность EMCDDA во многом основана на 
информационной сети Reitox, составленной из назначенных национальный институтов, 
ответственных за сбор данных и формирование отчетов по проблемам наркотиков и 
наркомании.  Эти институты носят название <<национальных точек фокуса>> или 
<<национальных пунктов наблюдения за наркоситуацией>>. Эта мониторинговая 
система охватывает 30 стран, причем сбор и обмен данными стандартизирован. В 
Миссии организации указано, что она призвана <<помогать определять направление 
наркополитик Евросоюза, и разрабатывать подходящие рекомендации странам для 
организации лечения, превентивных мер и деятельности по уменьшению вреда>>.

Среди других примеров информационных систем можно привести The Exchange on Drug 
Demand Reduction Action (EDDRA) --- база данных с доступом через Интернет, 
предоставляющая официальным ответственным лицам данные по программам снижения 
спроса на наркотики в Евросоюзе.

В Канаде функционирует Drug Treatment Court Information System, обеспечивающая 
деятельность т.н. <<наркосудов>>, которые являются типичным примером реализации 
политики снижения вреда.

Таким образом, по результатам краткого обзора, обнаруживается, что в странах, 
взявших на вооружение политику снижения употребления, среди ИС превалируют 
мониторинговые системы, отчасти --- аналитические, и на последнем месте --- 
медицинские и ресоциализирующие программы.
С другой стороны, в странах, придерживающихся политики снижения вреда, есть как 
мониторинговые системы, так и и системы, обеспечивающие различные социальные 
программы по снижению вреда.

\subsection{Зарубежный опыт прогнозирования распространения наркомании}

Рассмотрим подходы к прогнозированию наркоситуации с использованием эмпирических
данных (социологических опросов и др.) в странах Евросоюза и США.

В монографии центра EMCDDA дается обзор методов моделирования т.н. <<потребления
наркотиков с высокой степенью риска>> (high-risk drug use) или <<проблемного
употребления наркотиков>> (problem drug use) в Европе \cite{EMCDDA2001}. Описан
метод прогнозирования распространения наркомании с использованием ГИС
\cite{EMCDDA2001,Wiessing1999}, разработанный в рамках программы DIPEP --- Drug
Incidence \& Prevalence Estimation Program. Распространение наркомании
моделируется как процесс, схожий с заражением в эпидемиологии. В этой парадигме
ключевыми понятиями являются пути передачи инфекции, группы риска,
распространенность инфекции, её географическое распределение, эпидемический
цикл. Зоны распространения наркомании моделируются с помощью метода обратных
взвешенных расстояний. Таким образом моделируется изменение частоты и
распространенности потребления наркотиков в Великобритании.  На примере города
Глазгоу показана первоначальная концентрация <<наркоэпидемии>> внутри крупного
города с последующим распространением в окрестные города в течение пятилетних
циклов. 

Для оценки численности наркозависимых предлагается применение множества
динамических моделей, использующих доступные статистические данные. В данном
подходе основополагающим тезисом является необходимость уменьшить
неопределенность, связанную со скрытым характером процессов, характеризующих
наркоситуацию. Для этого применяются методы стратифицированных калибровочных
выборок, вероятностное моделирование, процессы Пуассона, метод множественного
захвата и перезахвата, оценка по связи распространенности тяжелых наркотиков и
частоте аквизитивных преступлений, марковские модели. Применение множества
методов и моделей для оценки одной целевой популяции позволяет более точно
судить о трендах и о порядке численности наркозависимых. 

В качестве альтернативной модели прогнозирования наркоситуации предлагается
модель с множеством индикаторов. В данной модели строится гипотеза о
существовании связей типа <<причина-следствие>> между индикаторами. Выделяется
три группы индикаторов: 
\begin{itemize}
    \item социальные;
    \item правовые;
    \item медицинские.
\end{itemize}
Авторами отмечается универсальность данной модели, возможность её использования
для исследования различных социально-экономических сценариев, адаптация модели к
исходным данным. Практическое применение данной модели позволило получить
неожиданные и контринтуитивные результаты. 

Отдельно европейскими исследователями выделяется класс динамических моделей.
Термин <<динамическая модель>> охватывает методы системного анализа и
моделирования временных рядов, применяемые для оценки распространенности
наркопотребления. Данные методы должны обеспечивать не только дескриптивный
анализ, но и моделировать процессы, лежащие в основе наркоситуации. Динамические
модели в отличие от статических описывают процессы во времени. Например,
исследуемым процессом может быть изменение состояния наркопотребителей в модели
<<поимка-мечение-повторная поимка>>. Модели поимки-повторной поимки делятся на
два класса: с открытой популяцией и с закрытой популяцией. В моделях с закрытой
популяцией предполагается, что популяция не изменяется на протяжении
исследуемого периода, а в моделях с открытой популяцией учитывается прибыль и
убыль особей популяции. Модели с закрытой популяцией проще и предназначены
скорее для анализа коротких временных периодов, но при достаточно большом
временном отрезке они, как правило, необъективны, и в таких случаях применяются
блее сложные модели с открытой популяцией.

В отчете Национального Института правосудия США \cite{TravisFeucht1995,
Chaiken1993} приводится анализ наркоситуации в 23 крупнейших городах США.
Датасет состоит из данных уриноанализа (на предмет позитивного тестирования на
кокаин, марихуану, опиаты) арестантов в разрезе по типам преступлений. Эта
информация, в частности, была использована для установления и объяснения связи
между наркотиками и преступностью. На основании снижения частоты позитивных
тестов на кокаин среди мужчин 15-20 лет строится гипотеза о продолжительном
снижении употребления кокаина по мере того, как представители данной возрастной
группы стареют.

В работе Дж. Колкинса \cite{Caulkins1995} производится оценка эластичности спроса 
на кокаин и героин на основании датасетов DUF и STRIDE, предоставленных
Национальным институтом правосудия США и Управлением по борьбе с наркотиками.
Автор пытается ответить на вопрос: насколько изменится потребление при
увеличении цен? В качестве метода математического анализа использованы
дифференциальные уравнения. Анализ показывает высокую эластичность спроса. В
качестве выводов автор связывает подъем употребления кокаина и героина в 1980-х
с существенным снижением цен на них. Как следствие, для проведения
антинаркотической политики могут быть полезными мероприятия, приводящие к
повышению цен на наркотики.

Датасеты ADAM и ADAM II --- результат опросов, проводимых с 1997 по 2013 годы в
35 (ADAM) и 10 (ADAM II) округах США по методу интервью с вопросами об
употреблении и участии в рынке наркотиков и уриноанализа арестантов в течение 48
часов с момента ареста на предмет обнаружения наркотических
веществ\cite{Hunt2013, Chapman2010}. Датасеты ADAM являются одними из
важнейших источников информации для лиц, определяющих наркотическую политику в
США, ввиду того, что они охватывают популяцию, мало представленную в иных
опросах. С точки зрения аналитических инструментов, для оценки трендов и
выявления годовых циклов применяются статистические методы, в частности, модель
логистической регрессии.  С помощью данной модели оценивается вероятность
положительного тестирования на содержание в моче определенного наркотика в
зависимости от типа преступления, сезона и года. В качестве вспомогательных
методов используются преобразования Фурье. Существенное внимание уделяется
заполнению пропусков во временных рядах, определению зависимых и независимых
переменных.

\newpage
\section*{Выводы по главе 1}
\addcontentsline{toc}{section}{Выводы по главе 1}

В главе был проведен обзор видов наркополитик, в общем виде описывающих
концептуальные основы и практические меры по урегулированию наркоситуации.
Исследована связь между постулируемыми в развитых странах стратегиями
антинаркотической политики и средствами их информационной поддержки. Обнаружено,
что в целом имеют тенденции к стандартизации в антинаркотической аналитике: от
проведения аналогичных опросов с определённой периодичностью до создания
филиалов крупных исследовательских центров во всех регионах страны,
отчитывающихся в центральный хаб с помощью стандартизированных отчетов.

В построении прогнозов имеются нерешенные проблемы с неоднородностью исходных
данных, требующие для своего решения продвинутых методов статистического
контроля данных и заполнения пробелов. Однако, сравнительная молодость
дисциплины способствует большому разнообразию применяемых методов, большая часть
которых заимствованы из других дисциплин, таких как эпидемиология, зоология,
эконометрика и др.

