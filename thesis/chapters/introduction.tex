В современном мире усиливается потребность в автоматизации процессов управления,
в частности административных процессов государственного уровня. Для этого
необходимо создавать информационные системы, отвечающие потребностям лиц,
принимающих решения. Управление социально-экономическим развитием города ---
комплексная задача, включающая в себя управление различными сферами и
подсистемами жизнедеятельности города с целью повышения его экономического,
творческого, научного потенциала, повышения качества жизни горожан, привлечения
инвестиций, создания комфортных условий для предпринимательства, обеспечения
социальной защищённости и благоприятного эмоционально-психологического климата в
городе.

Одним из проблемных явлений в жизни общества является проблема наркомании,
наркотической зависимости. Наркомания наносит ощутимый экономический ущерб
обществу. Но не менее опасен урон общественной морали и принципам благоразумия 
потому, что наркотики разрушительно влияют на мозг человека. А ведь ещё Рене
Декарт провозгласил классический прицип рационализма Нового времени: <<Cogito
ergo sum>>, или <<Мыслю, следовательно существую>>. Таким образом, можно
предположить, что наркотизм причиняет человеку поистине экзистенциальные
страдания. Но несмотря на чисто гуманистические соображения о стремлении помочь
ближнему нужно понимать, что наркотизм --- сложный феномен, полностью победить
который не смогли ещё ни в одной стране мира.

Характерными чертами наркоситуации являются скрытность, обусловленная
её нелегальной природой, сетевой характер распространения, рискованность, но и
прибыльность наркорынка, наличие связей между наркоситуацией и другими
негативными явлениями: безработицей, коррупцией, преступностью и др. Все эти
особенности требуют подходов, которые способны были бы адресовать их.

Общепринятым подходом к организации антинаркотических программ является
использование наукоемкой аналитики для поддержки принятия управленческих решений
ответственными лицами. И здесь большую \textbf{актуальность} имеют модели,
способные не только описать свойства процесса, но и объяснить внутренние
механизмы его протекания, связь наркотизации с другими сферами города,
генерировать т.н.  <<actionable insights>> --- идеи, которые позволяют
действовать: применять конкретные рычаги управления, выявленные при помощи
моделирования, для влияния на наркоситуацию. Более того, интересны модели,
способные к самообучению, универсальной подстройке под исходные данные.

Однако, в практике повседневного информационно-аналитического обеспечения
органов государственной власти в Санкт-Петербурге, в основном, применяются
традиционные модели прогнозирования, известные по эконометрике, математической
статистике: модель авторегрессии с интегрированным скользящим средним, модель
Хольта-Винтерса, модель линейной регрессии и др. Данные модели для своего
корректного применения требуют наложения определённых ограничений на исходные
данные, таких как их линейность, отсутствие т.н. outliers --- случайных
всплесков, наличие сезонных циклов и т.п. В то же время, данные, наблюдаемые при
мониторинге наркоситуации, зачаствую являются нелинейными, неполными,
полученными из экспертных или вовсе неформальных источников. Для учета этих
особенностей могут подходить интеллектуальные методы прогнозрование, одним из
которых является нечеткая логика. Нечеткую логику зачастую применяют вместе с
алгоритмами машинного обучения, такими, как искусственные нейронные сети и др.
Такие авторы, как Н. А. Абдуллавева \cite{Abdullaeva2010}, М. Г. Мамедова, З.Г.
Джабраилова \cite {Mamedova2005}, П. С. Пак, Г. Ким \cite{Pak2005}, А. Сасу
\cite{Sasu2010} демонстрируют, хоть местами и противоречивые, но обнадеживающие
результаты в применении интеллектуальных методов и нечеткой логики для
прогнозирования социальных процессов. 

\textbf{Практическая значимость работы.} Подход к прогнозированию с
использованием методов интеллектуального анализа временных рядов, гипотетически,
может эффективно решить задачу государственного управления и оценки
стратегического развития региона в области борьбы с наркоманией. Для этого
соответствующее программное обеспечение должно быть внедрено в действующие
информационно-аналитические системы. Данныая работаа выполняется по заказу
Санкт-Петербургского информационно-аналитического центра, деятельность которого
связана с разработкой информационно-аналитических систем поддержки принятия
решений в органах государственной власти и местного самоуправления.

\textbf{Цель ВКР} состоит в повышении эффективности поддержки принятия решений
при прогнозировании развития наркоситуации с Санкт-Петербурге. Для достижения
поставленной цели необходимо решить следующие задачи:
\begin{itemize}
\item Обзор существующих в мире политик и моделей организации антинаркотической
деятельности на государственном уровне;
\item Описание используемой в настоящее время системы мониторинга и анализа
наркоситуации в Санкт-Петербурге;
\item Обзор методов и моделей прогнозирования временных рядов;
\item Разработка и опытная проверка метода прогнозирования на основе нечеткой
логики, выработка рекомендаций к применению;
\item Разработка компонента государственной информационной системы, реализующего
новый метод прогнозирования.
\end{itemize}



